\chapter{Introduction}\label{chap:Introduction}

\section{Background}

Multiple sequence alignment (MSA) analysis, a fundamental approach in molecular biology, is required for nearly all comparative sequence studies, including evolutionary, functional and structural aspects. A multiple sequence alignment is usually a matrix, in which each row represents a sequence, and each column corresponds to the equivalent positions across all sequences aligned. Each element of the matrix contains a symbol, which can be either `-' (a gap) or a sequence symbol (an amino acid for DNA/RNA sequences, or a nucleotide base for protein sequences). \cite{Edgar:2006aa}

Due to the size and complexity of the multiple sequence alignment datasets, computer visualization techniques are essential to assist in understanding, analyzing and evaluating the alignment results. Numerous MSA visualization applications, either stand-alone or web-based, have been developed in recent decades, ranging from simple MSA viewers, to complex editing and analysis platforms. Many of them have graphical user interfaces showing a grid of symbols, usually along with various coloring, sizing and annotation. \cite{Procter2010aa}

\section{Related Work}

Coloring an alignment has been widely implemented to highlight specific regions, properties or patterns. \cite{Procter2010aa} The simplest way of coloring is to use a fixed color scheme, in which each sequence symbol has its own color, and will not be changed across the alignment. The assignment of color schemes are usually based on some empirical properties or chemical classifications. There are numerous color schemes available for different purposes, but Taylor \cite{LIN2002361} and Clustal \cite{Thompsonaa} are widely used and often considered de facto standards.

However, using pre-determined color schemes to color sequence symbols, either amino acids or nucleotide bases, are not always helpful to phylogenetic analysis. Those fixed schemes focus on chemical differences between specific types of symbols, rather than between sequences, regions, blocks or structures of the alignment. For example, if symbols are examined within a certain column, the colors reflect the similarities fairly well. However, if the alignment is analyzed horizontally, comparative information within a row or a rectangular region are unable to be displayed, like how one sequence is similar with others, which section aligns better than another, which subset of sequences in which range of columns are more conserved, or how robust is the alignment accuracy.

Here we suggest a new MSA visualization tool, Mavis (Multiple Alignment VISualization), which focuses on highlighting evolutionary relationships and internal structures of an alignment. Instead of relying on chemical properties or any other information from symbol types, our approach takes only symbol-symbol correlation into consideration. The most valuable result we would like to provide is a whole picture of the MSA.

\section{Thesis Outline}

This thesis is organized in the following way: Chapter 2 briefly summarizes how Mavis processes data input, creates colors and displays the colored MSA graph. Chapter 3 explains the algorithm step by step, from scaling to mapping and optimization. Chapter 4 describes our implementation of the algorithm, including the back-end, API and front-end. Chapter 5 presents test results to show the validity and performance of our implementation. The whole work is discussed and concluded in Chapter 6.