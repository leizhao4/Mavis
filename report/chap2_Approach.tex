\chapter{Approach}\label{chap:Approach}

In this thesis, we address a different method to visualize multiple sequence alignments. The main objective is to correlate alignment similarity (or confidence) with color similarity. In the alignment matrix, a well-aligned area forms a solid-color shape, while a poor-aligned one becomes a mosaic of various colors. Alignment-level information, such as evolutionary relationships, phylogenetic characteristics and alignment accuracy should be revealed by examining the graph generated by our tool.

\section{Data Input}

Unlike most traditional coloring approaches, we start with an set of pairwise scores, rather than the alignment itself. Each of these scores represents the similarity or alignment confidence between two symbols aligned together in the same position, or, in the same column. Every pair of symbols in the same column has a score.

This set of data can be generated by GUIDANCE \cite{Penn:2010aa,Penn:2010ab}, a measure quantifying alignment uncertainty. It produces a confidence score between 0 and 1, named GUIDANCE score, for each symbol, symbol pair, column and sequence in the alignment. The symbol-pair GUIDANCE confidence scores can be used in our study to visualize MSAs.

\section{Color Generating}

The colors of the symbols are generated in such way that the color differences represent symbol differences. We consider the confidence scores reversely, as distances, and convert all the scores of a column into a $N*N$ distance matrix, where $N$ is the number of symbols. These $N$ symbols can be represented by $N$ points in at most $N-1$ dimensions, such that the Euclidean distance between any pair of points is exactly equal to the distance in the matrix. Once we can map these points to a color space, those corresponding colors should be able to fulfill our requirements.

However, most color spaces are three-dimensional or under. It is necessary to reduce the dimension from $N-1$ to at most 3 in order to process the mapping, but still preserve the distances as much as possible. This dimension reduction is done by a statistical technique called \emph{classical multidimensional scaling} \cite{Borg:1997aa}.

\section{Color Optimization}

Up to now, our approach only operates on the column level. Further adjustment is still needed to achieve the alignment-level objectives. We notice that the color space can be arbitrarily rotated and/or flipped, without changing the distance between any pair of symbols. For instance, all reds and all greens are virtually identical in the sense of distances and quality of alignment. Converting them into the same color pattern will better support this idea and remove unnecessary color noises.

We defined a penalty function to quantify the overall noise level of an alignment. The ideal combination of rotating and flipping each column should result in the lowest penalty value. For large MSAs, finding the global optimum is not always possible, and the approximate solution is also an acceptable option.

\section{MSA Visualization}

We choose to implement a web-based user interface, because of its flexibility, portability and light weight. Many features are provided to enhance the visualization effects, such as sorting sequences by their average colors, toggling to show/hide symbol names, and multiple-MSA support. An API between the front-end and the back-end is designed to make the system more extensible.